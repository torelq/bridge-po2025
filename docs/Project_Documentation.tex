\documentclass{article}
\usepackage[utf8]{inputenc}
\usepackage[T1]{fontenc}
\usepackage{lmodern}
% \usepackage{amssymb, amsmath, amsthm}
\usepackage{csquotes}
\usepackage{graphicx}

\setlength{\parindent}{0pt}
% \newcommand{\hi}[1]{\textit{#1}}
% \newcommand{\imp}{\Rightarrow}
% \newcommand{\limp}{\Leftarrow}
% \newcommand{\mbb}[1]{\mathbb{#1}}
\newcommand{\ch}[1]{\texttt{#1}}

\title{Dokumentacja Projektu \enquote{Brydż}}
\author{Hubert Krata}
\date{PO 2025}

\begin{document}
\maketitle

\section{Opis}

Program pozwala czterem użytkownikom na grę w brydża przez LAN. Program obsługuje licytację i liczenie punktów z wielu partii. Program posiada \enquote{debug mode}, ułatwiający demonstrację/debugging.

\section{Ogólna architektura}

Projekt zrealizowaliśmy w architekturze MVC.\\

Opis komponentów, którym odpowiadają pakiety:
\begin{itemize}
	\item \ch{model} --- zawiera klasy modelujące logikę gry (w szczególności \ch{Game}, \ch{Bidding} i \ch{Scoring})
	\item \ch{controller} --- zawiera logikę działania klienta (event handlery GUI, odbieranie wiadomości, itp). Jeśli klient hostuje serwer, trzyma referencję do obiektu \ch{Server}, ale nic z nią nie robi.
	\item \ch{view} --- zawiera klasy odpowiadające za wygląd GUI. Zastosowaliśmy framework JavaFX.
	\item \ch{server} --- jedyną publiczną klasą jest \ch{Server}, odpowiadająca za\ldots serwer.
	\item \ch{communication} --- klasy ułatwiające komunikację klient-serwer.
\end{itemize}

\section{Nieco bardziej szczegółowe opisy niektórych komponentów}

\subsection{\ch{Model}}

Podstawowe klasy modelu:
\begin{itemize}
	\item \ch{Bidding} --- odpowiada za licytację.
	\item \ch{Game} --- odpowiada za przebieg rozdania (w szczególności, zrzucanie kolejnych lew). Zawiera referencję do \ch{Bidding}.
	\item \ch{Scoring} --- odpowiada za liczenie wyniku z wielu gier.
\end{itemize}

Istnieją testy jednostkowe do modelu.

\subsection{\ch{Communication}}

Dzięki pakietowi \ch{communication}, klient (\ch{Controller}) i serwer (\ch{Server}) nie komunikują się za pomocą \enquote{gołych} socketów i bajtów. Zawiera dwa podpakiety: \ch{messages} i \ch{streams}.

\subsubsection{\ch{Messages}}

Podstawą komunikacji są wiadomości. Komunikacja odbywa się jednocześnie na dwa sposoby: request-response (\ch{xxRequest}-\ch{xxResponse}), gdzie klient pyta o coś serwer, a serwer odpowiada, oraz poprzez wiadomości \ch{xxNotice}, które wysyła serwer do klientów, informując ich o jakiejś zmianie.\\
Całość stanu gry jest jawna dla każdego klienta. Przesyłane są zmiany stanu, chociaż jest możliwość poproszenia o cały stan.\\

Wiadomości są silnie otypowane jako \ch{ClientToServerMessage} lub \ch{ServerToClientMessage}.

\subsubsection{\ch{Streams}}

Streamy obudowują komunikację, tak aby można było przesyłać wyłącznie wiadomości z kontrolą typów oraz umożliwiając abstrakcję na komunikację przez socket albo lokalnie.\\

Inne pakiety powinny korzystać \textit{wyłącznie} z odpowiednio \ch{ClientMessageStream} lub \ch{ServerMessageStream} opakowujących inne rzeczy, np.:\\
\ch{new ClientMessageStream(new TCPMessageStream(new Socket(...)))}\\

\ch{\{Client, Server\}MessageStream} może opakowywać \ch{TCPMessageStream} opakowujący \ch{Socket} albo \ch{PipedMessageStream} opakowujący lokalny pipe. Chociaż ostatecznie \ch{PipedMessageStream} nie jest używany, był przydatny do testów.\\

Dziwna nieco architektura jest konsekwencją chęci zachowania kontroli typów wiadomości.

\subsection{\ch{Server}}

Serwer jest zbudowany jako samowystarczalny. \ch{Controller} musi zrobić:\\
\ch{Server server = new Server();\\
Thread serverThread = server.runInNewThread();\\
// ...\\
serverThread.join();}\\
I nie musi się serwerem więcej przejmować.\\

Serwer ma trzy główne części:
\begin{itemize}
	\item \ch{acceptorTh} --- wątek nasłuchujący na \ch{ServerSocket}, akceptujący każdego klienta i dodającego go do listy klientów z utworzeniem właściwych mu wątków/struktur danych.
	\item klienci --- każdy klient ma dwa wątki: jeden pisze na \ch{ServerMessageStream}, drugi zeń czyta i wrzuca wiadomości na kolejkę wydarzeń. W razie przepełnienia buforów, klient jest odłączany.
	\item \ch{mainLoopTh} --- sekwencyjnie przetwarza kolejne rządania/wydarzenia z kolejki i wrzuca odpowiedzi na bufory klientów. Przechowuje stan gry.
\end{itemize}

Testy serwera to nie testy jednostkowe, a małe programy, które pomogły przy debugowaniu.

\section{Napotkane trudności i uwagi do architektury}

\subsection{Napotkane trudności}

Trudnym bugiem był problem z serializacją obietu \ch{Game}. Obiekt \ch{Player} rozszerzał \ch{Hand} i implementował \ch{Serializable}, ale \ch{Hand} nie implementowało \ch{Serializable}, przez co nie przesyłał się stan ręki.\\

Oprócz tego, problem sprawiało rozspójnianie stanu.

\subsection{Przemyślenia dot. architektury}

Architektura z pewnością nie jest doskonała. Część decyzji była pewnie błędna, albo przynajmniej nieoptymalna. Niemniej, równie ważne co polepszanie architektury, jest to, aby wiedzieć, kiedy przestać. Obecny kod spełnia założenia projektu i opiera się na poprzednich decyzjach, a refactoring kosztuje.

\end{document}